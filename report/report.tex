\documentclass{article}

\usepackage[margin={1in,1in}]{geometry}
\usepackage{float}
\usepackage{amsmath}
\usepackage{listings, color}

\begin{document}

\title{CS 118 - Project 1}
\author{Alex Crosthwaite -- Jacob Nisnevich -- Jason Yang}

\maketitle

\section{Design}

From a top-level perspective, we implemented four different classes, utilizing object-oriented abstractions, to create the web client and web server. These include the following classes: \texttt{HttpRequest}, \texttt{HttpResponse}, \texttt{Client}, and \texttt{Server}. In the following sections we will describe our high-level design decisions in implementing each of these classes.

\subsection{HTTP Request and Response}

For the HTTP request and response abstractions we choe to make separate classes for each message, with slight differences. Both \texttt{HttpRequest} and \texttt{HttpResponse} have \texttt{encode} and \texttt{consume} methods that encode and decode the HttpRequest string respectively. \\

\noindent
In both cases, the \texttt{consume} methods take in an encoded request or response string and parse it to the appropriate class member variables. These functions work in two steps: first splitting the string by new lines and then using \texttt{std::regex} to parse the first line and then each of the following header lines. \\

\noindent
The class differ in how they treat the header fields however. For the \texttt{HttpRequest}, each line of the request string following the request line is parsed to an \texttt{unordered\_map} of header name to header value. For the \texttt{HttpResponse}, all header fields other than the \texttt{Content-length} header are ignored due to the implementation of the web client. As such, no map data structure with an ambiguous number of headers is required.

\subsection{Web Client}

The web client class, \texttt{Client}, is instantiated in the \texttt{web-client.cpp} file, which takes as parameters one or more URLs. Before being passed to the \texttt{Client} class, each of the URL arguments is parsed to a \texttt{url\_t struct} consisting of host, port, and file path. Each of these \texttt{url\_t}'s is then added to a \texttt{map} of host-port pairs to file paths. This map is created because each individual host-port pair corresponds to a unique socket, allowing for multiple file requests to the same host-port pair to utilize HTTP/1.1 persistent connections. \\

\noindent
The \texttt{Client} class has one constructor that takes two parameters: the host-port to file path map and the number of URLs. Using the second parameter, the client decides whether to use HTTP/1.0, for a single request, or HTTP/1.1, for multiple requests. Then, for each host-port pair it creates a socket and initializes a connection. For each file path in the vector of file paths for the host-port pair, the client sends an HTTP request to the server, waits for a response, and writes it to a feile, assuming the response had a 200 status code. Note that our implementation of HTTP/1.1 persistent connections does \textit{not} use pipelines.

\subsection{Web Server}

\section{Problems and Solutions}

\subsection{Client File Reception}

Problem: How does the client know the entire file has been transmitted \\

\noindent
Solution: use content length \\

\subsection{Client Multiple URL Handling}

Problem: When parsing multiple URLs with muliple host, port, file combinations, how do we structure our data. \\

\noindent
Solution: Use a map from host-port pairs to file path vectors

\section{Build Instructions}

For the most part, we did not modify the Vagrantfile or Makefile. However, we did add two lines to the Vagrantfile:

\begin{lstlisting}
sudo add-apt-repository ppa:ubuntu-toolchain-r/test
...
sudo apt-get install -y g++-4.9
\end{lstlisting}

\noindent
These two lines add the g++-4.9 repository and then installs the new edition of g++. Our implementation of the client and server required this version of g++ in order to use \texttt{std::regex} in parsing HTTP requests, responses, and URLs.

\section{Test Cases}

\section{Contributions}

\subsection{Alex Crosthwaite}

\begin{itemize}
	\item Server (50\%)
\end{itemize}

\subsection{Jacob Nisnevich}

\begin{itemize}
	\item Client (50\%)
	\item HTTP Request and Response Classes
\end{itemize}

\subsection{Jason Yang}

\begin{itemize}
	\item Server (50\%)
	\item Client (50\%)
\end{itemize}

\end{document}
