\documentclass{article}

\usepackage[margin={1in,1in}]{geometry}
\usepackage{float}
\usepackage{amsmath}
\usepackage{listings, color}

\begin{document}

\title{CS 118 - Project 1}
\author{Alex Crosthwaite -- Jacob Nisnevich -- Jason Yang}

\maketitle

\section{Design}

From a top-level perspective, we implemented four different classes, utilizing object-oriented abstractions, to create the web client and web server. These include the following classes: \texttt{HttpRequest}, \texttt{HttpResponse}, \texttt{Client}, and \texttt{Server}. In the following sections we will describe our high-level design decisions in implementing each of these classes.

\subsection{HTTP Request and Response}



\subsection{Web Client}

\subsection{Web Server}

\section{Problems and Solutions}

\subsection{Client File Reception}

Problem: How does the client know the entire file has been transmitted \\

\noindent
Solution: use content length \\

\subsection{Client Multiple URL Handling}

Problem: When parsing multiple URLs with muliple host, port, file combinations, how do we structure our data. \\

\noindent
Solution: Use a map from host-port pairs to file path vectors

\section{Build Instructions}

\section{Test Cases}

\section{Contributions}

\subsection{Alex Crosthwaite}

\begin{itemize}
	\item Server (50\%)
\end{itemize}

\subsection{Jacob Nisnevich}

\begin{itemize}
	\item Client (50\%)
	\item HTTP Request and Response Classes
\end{itemize}

\subsection{Jason Yang}

\begin{itemize}
	\item Server (50\%)
	\item Client (50\%)
\end{itemize}

\end{document}
